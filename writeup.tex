\documentclass{acm_proc_article-sp}
\begin{document}

\title{Monoids and their uses}
\subtitle{[Extended Abstract]
\numberofauthors{1}
\author{
\alignauthor
Mateusz Kowalczyk
       \affaddr{University of Bath}\\
       \email{mk440@bath.ac.uk}
       \email{fuuzetsu@fuuzetsu.co.uk}
}

\date{1 May 2013}

\maketitle
\begin{abstract}
In light of the recent increase in interest towards functional
elements of programming, this report aims to provide an overview of
monoids. A mathematical definition of a monoid is given followed by
some examples of common monoids and examples of usage of monoids in
every day programming environment.
\end{abstract}

\keywords{monoid, functional programming}

\section{Introduction}
There has recently been a trend where more and more programmers are
becoming interested with functional programming languages. This is not
a global change where we'll see Haskell or Scheme as the most popular
language in the world by the end of 2013. In fact, no
purely-functional languages even make the top 20 in popularity
rankings\cite{tiobe:rankings}. Many of the languages that are gaining
popularity nowadays at least include functional programming
elements. Some examples include C# (popular LINQ library is
implemented with monads), JavaScript (first-class functions and
closures) and Python (\lambda, first-class functions, built-in
functional programming libraries\cite{python:itertools}). With
increased interest in the functional part of the languages that many
programmers already use, it's natural that interest in type theory and
category theory is increased as that's often the basis for functional
programming languages.

In light of this, I'm going to talk about a simple category of monoids
and some sample examples of practical uses of monoids in
software. I'll also briefly touch upon where monoids lie in terms of
category theory.

This report uses Haskell as the language used for demonstration of any
code where code is present, unless stated otherwise.

\section{Monoids}
There are a couple of definitions that can be given for what a monoid
actually is. Monoids play a part in abstract algebra and category
theory. From a point of view of abstract algebra, a monoid is a simple
algebraic structure with three axioms. A monoid is a set $S$ and a
binary operation \cdot such that:
\begin{itemize}
\item Closure
  \begin{itemize}
  \item $\forall x,y \in S: x \cdot y \in S$
  \end{itemize}
\item Associativity
  \begin{itemize}
  \item $\forall x,y,z \in S: (x \cdot y) \cdot z = x \cdot (y
    \cdot z)$
  \end{itemize}
\item Identity
  \begin{itemize}
  \item $\exists e \in S: \forall x \in S: e \cdot x = x \cdot e
    = x$
  \end{itemize}
\end{itemize}
A definition of a monoid in category can also be given and is
probably the more interesting entity to study when put in terms of
category theory. In this report however, the definition of the monoid
as an algebraic structure will suffice. From here on, $mappend$ will
refer to the binary operation for any monoid and $mempty$ to its
identity element.
\subsection{Examples of structures forming a monoid}
There is a great amount of structures in use in mathematics that form
a monoid under a certain operation. In fact, many of these are already
indirectly studied as monoid naturally forms a semigroup with an
inverse. It's also worth noting that a group forms a monoid with an
inverse. This means that monoids are not a fresh or a totally alien
concept to mathematicians.

\subsubsection{Natural numbers under addition}
A fairly trivial example of a monoid is a set of all natural numbers
$\mathbb{N}$ and a binary operation $+$:
\begin{itemize}
\item Closure
  \begin{itemize}
  \item $\forall x,y \in \mathbb{N} : x + y \in \mathbb{N}$
  \end{itemize}
\item Associativity
  \begin{itemize}
  \item $\forall x,y,z \in \mathbb{N} : (x + y) + z = x + (y + z)$
  \end{itemize}
\item Identity
  \begin{itemize}
  \item $\forall x \in \mathbb{N} : 0 + x = x + 0 = x$
  \item $0 \in \mathbb{N}$
  \end{itemize}
\end{itemize}
In this instance, by property of commutativity of $+$ on integers, we
can also see that $x + y = y + x$. Monoids with this property are
called commutative monoids. Here $mappend = (+)$ and $mempty = 0$

\subsubsection{Lists under concatenation}
A bit less trivial example is an example of lists under concatenation.
We use \texttt{x :: a} to mean $x$ is a member of (or a proof of) the
type $a$, for any $x$ and any $a$. Denoting lists as with square
braces, we use \texttt{xs :: [a]} to mean $xs$ is a member of type
$[a]$ where $[a]$ is a list of elements of type $a$. Note that this
holds even if $xs$ is an empty list: $[]$. We now define $++$ to be a
binary operation that is used to append two lists:
\texttt{
(++) :: [a] -> [a] -> [a]
[] ++ ys = ys
(x:xs) ++ ys = x:(xs ++ ys)
} where $:$ is cons operation and pattern matching is used.

To form a monoid under any $x :: [a]$ with $++$, it's possible to show
that for any set $L$ with all elements of type $[a]$:
\begin{itemize}
\item Closure
  \begin{itemize}
  \item $\forall x,y \in \in L: x ++ y \in L$
  \end{itemize}
\item Associativity
  \begin{itemize}
  \item $\forall x,y,z \in L: (x ++ y) ++ z = x ++ (y ++ z)$
  \end{itemize}
\item Identity
  \begin{itemize}
  \item $\forall x \in L: [] ++ x = x ++ [] = x$
  \item $[]$ is the identity
  \end{itemize}
Proof by structural induction on $++$ is possible and fairly easy but
is omitted for brevity. Here $mappend = (++)$ and $mempty = []$

This monoid is more interesting than the natural numbers over addition
example because it works for any arbitrary list of polymorphic type
$a$. As a consequence to this, we are now able to append/concatenate
anything that we can directly represent as a list. An example is type
$String$ which can simply treat as an alias of $[Character]$ where $Character$
is a concrete type that we can use to represent letters, numbers
etc. Given $x = "Hello " :: String$ and $y = "World!" :: String$, we
can exploit use our binary operation as $x ++ y = "Hello World!"$. For
illustration purposes, any $c :: [Character]$ will be rendered as a
human-readable string, so we will write ``Hello'' instead of ``['H',
'e', 'l', 'l', 'o']''.

Last thing to note that unlike the natural numbers example, this
monoid isn't commutative. $[2, 3] ++ [4, 5] \neq [4, 5] ++ [2,
3]$. This is interesting because we can form a whole new monoid, dual
to the existing one simply by making our binary operation treat the
arguments in reverse order. That is, we apply
\texttt{
flip :: (a -> b -> c) -> b -> a -> c
flip f x y = f y x
}
to the binary operator of a non-commutative monoid to get a whole new
operator out and a new monoid as a consequence. It can be shown that
the monoid laws still hold.

\section{Uses}
Having talked about what monoids are and giving two examples, we can now
demonstrate some uses for practical situations.

\subsection{Logging}
It is often desirable that during an execution of a function, we
gather some data along the results. In a case of a simple iterative
function, this task would probably be achieved by appending to a
separately defined list after each operation. This solution does not
work if we have a recursive function however.
In both cases, we have some function $f :: a -> c$ where $a$ is
the type of argument, $c$ is the type of the result. With logging, we
also want to get some data out of the way. For convenience, I'm going
to define a \texttt{Pair a b} type that takes any value of type $a$ and any
value of type $b$ and stores them. So given $x :: String$ and $y ::
Integer$, we can form a pair with \texttt{pair x y :: Pair String
Integer}. We then define \texttt{fst (Pair x y) = x} and \texttt{snd (Pair x y) =
y}. With that in place, we want to change our $f :: a -> c$ to $g :: a
-> Pair b c$ where $b$ is the type of the log we want to
produce. As an example, consider a factorial function $fact$:
\texttt{
fact :: Natural -> Natural
fact 0 = 1
fact x = x * (fact (x - 1))
}
This works just fine (albeit very inefficiently) as is but what if we
want to also get a string at each step saying what $x$ was? Assume
that we have a function $show :: Natural -> String$.
\texttt{
fact :: Natural -> Pair [String] Natural
fact 0 = pair [``x is 0''] 1
fact x = pair [``x is '' ++ x] (x * (fact (x - 1))
}
Very quickly we'd find out that this doesn't work: $x * (fact (x -
1))$ can not work as $fact$ now returns $Pair [String] Natural$ and
not $Natural$! We can solve this with the list monoid in a fairly
elegant matter.
\texttt{
fact :: Natural -> Pair [String] Natural
fact 0 = pair [``x is 0''] 1
fact x = pair ([``x is '' ++ x] ++ (fst resultPair)) (x * (snd
resultPair))
  where resultPair = fact (x - 1)
}
This way we simply fetch the result of $fact (x - 1)$ and then
concatenate the log together. Thanks to the associative property of
monoids, we're guaranteed that the log will come back in the correct
order. We can even have the log come back in the reverse order by
using the dual of the list monoid, by using $flip (++)$ instead of
$++$. Even better, the log can be anything that forms a monoid! Here
we use a list monoid (inside another list) but we can use this pattern
to keep a list of anything that forms a monoid, simply by
substituting $++$ for the binary operation of that specific
monoid.\footnote{We can take this even further by making our $Pair$ a
  monoid itself and then using that as the log. For any monoid $a$ and
any monoid $b$, we can make $Pair a b$ a monoid by making defining
$mempty = Pair (mempty a) (mempty b)$ and $mappend (Pair a b) (Pair c
d) = Pair (mappend a c) (mappend b d)$.} In
fact, this is precisely what Haskells $Writer$ monad does.

\subsection{Folding}
Another use of monoids is folding. Folding a $[a]$ means applying a
certain function $f :: a -> b -> a$ to an accumulator and each element
of the list. We can define a left fold\footnote{In addition to the
  left fold, we could also have picked the right fold which starts to
  apply the function from the end of the given list. Due to
  associativity of monoids, left and right folds using a monoid
  operation have exactly the same semantics.} as follows:
\texttt{
foldl :: (a -> b -> a) -> a -> [b] -> a
foldl f acc [] = acc
foldl f acc (x:xs) = foldl (f acc x) xs
}
It's actually possible to define this fold operation for all
monoids. Even better, we get it for free from the definition of the
monoid!
\texttt{
mconcat (Monoid a) => [a] -> a
mconcat = foldl mappend mempty
}
Here \texttt{(Monoid a) => [a] -> a} denotes that a has to be a
monoid. We can now use mconcat on any monoid: $mconcat ["Hello ",
"everyone ", "!!"] = "Hello everyone here!!"$ and $mconcat [1, 2, 3] =
6$. The uses for folding are far and wide in the programming world:
calculating sums, products, means, finding minimums/maximums etc. There is an
incredible amount of operations that can be done with a fold, even
including sorting and finding primes\cite{haskell:fold}. It's
therefore quite convenient that for any monoid we can form, we always
have a free folding operation on it. While a list monoid might not
have the most exciting fold, all the programmer is limited by is his
ability to form monoids. Of interest to many people might be the fact
that trees are themselves monoids in a similar fashion that lists are
and trees are and both these constructs are very widely used in programming.

It has recently been suggested that monoids can be used to design more
efficient and elegant algorithms for the MapReduce framework, even
going as far as automating the process\cite{Lin:monodify}

\section{Closing words}
I don't believe that any have touched upon monoids or any closely
related topics. In fact, I think that the closest that someones talk
has come to the topic was the ``Semantics of Lazy Evaluation'' talk on
$30^{th}$ of May, simply through the fact that lazy evaluation is
mostly found in functional programming languages. My own talk simply
touched upon what a monoid is in algebraic terms and gave a few
examples. As show during the talk, it's quite possible to shoehorn
monoids (and many other structures) onto imperative languages but it
requires large amount of boilerplate code. Oftentimes people find it
worth to do anyway and libraries encapsulating various functional
programming (and by proxy, category theory and often type theory,
depending on the language) concepts are written and
published\cite{java:functional}.

I don't believe that monoids have had a huge impact on foundations of
computation, both on how we compute and how we think about
computation. Having said that, I'd like to make clear that I'm not
saying monoids aren't useful or that they are aren't worth studying. I
simply mean to convey that monoids are just a part of a bigger
picture. Should we not study monoids because we have more powerful
monads? Should we not study semigroups because we have groups? Clearly
this isn't the case and very often we can't form a monoid or a group
and we need slightly weaker structures/categories. Of course, any
research done to monoids also applies to everything that is isomorphic
or homomorphic to the monoid category also benefits. It's therefore
important to not abandon the study of monoids because they seem like
such a simple structure. As Donald Knuth himself says, there is still
low hanging through in Computer Science and many, many things haven't
yet been properly studied, even though they might have been around for years.


\subsection{References}
\bibliographystyle{abbrv}
\bibliography{writeup}  % sigproc.bib is the name of the Bibliography in this case

\balancecolumns

\end{document}

%%% Local Variables:
%%% mode: latex
%%% TeX-master: t
%%% End:
